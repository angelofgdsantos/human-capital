\section*{abstract}

This paper studies the role of financial incentives in teacher's absteseeism and its 
relationship with learning for schools in India. Using a randomized experiment with parateacher, 
wich monitored them daily and increased salries conditional on days worked,
the paper estimates the reduced form impacts of the program on teachers abseteeism and learning. 
In addition, a structural dynamic labor supply model is used to understand the mechanisms 
and compute optmal policy contrafactual. The results indicates that teachers strongly respond
to financial incentives and the optmal policy could be relatively more cost-effectve.

\section*{Introduction}

In 2003, an NGO called Seva Mandir ran an experiment with paralegal teachers in order to 
increase teacher's assiduity rate in indian schools. Teacher's absenteeism is relatively high
in the country, a national survey found that 24 \% of the teachers were missing teaching hours. 
The world Bank reported india among the 3 countries that suffer the most with abseentesim
with Uganda and Kenya. Teachers a


